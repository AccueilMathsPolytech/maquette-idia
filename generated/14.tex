\section{UE Industrialisation de la production logicielle}%
\label{sec:UEIndustrialisationdelaproductionlogicielle}%
A COMPLETER%
\subsection{Outils pour le développement logiciel }%
\label{subsec:Outilspourledveloppementlogiciel}%

%
A COMPLETER%
\begin{longtable}{c c c c c c c}%
\hline%
Cours&TD&TP&Projet&Eval&Personnel&Total\\%
nan h&nan h&nan h&nan h&nan h&nan h&nan h\\%
\hline%
\end{longtable}%
\subsection{Méthodes et outils devops}%
\label{subsec:Mthodesetoutilsdevops}%

%
L'élément constitutif Techniques de Devops permet d’explorer l’écosystème d’outils utilisés dans le développement d'un logiciel, en comprendre les principes et pouvoir les configurer. Ces outils visent à automatiser la production du système final à partir des différents artefacts produits (code source) ou utilisés, estimer ou garantir la qualité, permettre à différentes personnes de travailler simultanément sur le même artefact, documenter l'évolution des artefacts au cours du temps, etc.%
\begin{longtable}{c c c c c c c}%
\hline%
Cours&TD&TP&Projet&Eval&Personnel&Total\\%
nan h&nan h&nan h&nan h&nan h&nan h&nan h\\%
\hline%
\end{longtable}%
\subsection{Parallélisation de données}%
\label{subsec:Paralllisationdedonnes}%

%
A COMPLETER%
\begin{longtable}{c c c c c c c}%
\hline%
Cours&TD&TP&Projet&Eval&Personnel&Total\\%
nan h&nan h&nan h&nan h&nan h&nan h&nan h\\%
\hline%
\end{longtable}%
\subsection{Total UE :}%
\label{subsec:TotalUE}%

%
\begin{longtable}{c c c c}%
\hline%
Travail maquette&Travail personel&Travail total&Crédits ECTS\\%
nan h&nan h&nan h&5\\%
\hline%
\end{longtable}