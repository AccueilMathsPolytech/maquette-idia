\section{UE Industrialisation de la production logicielle}%
\label{sec:UEIndustrialisationdelaproductionlogicielle}%
A COMPLETER%
\subsection{Outils pour le développement logiciel }%
\label{subsec:Outilspourledveloppementlogiciel}%

%
En s'appuyant sur la connaissance d'un ou deux langages informatiques, la matière aborde quelques points complémentaires relative à la qualité et à la construction : le preuve de programme, l'analyse statique, la gestion des dépendances, les bonnes pratiques sur le développement collaboratifs (prise de recul sur les usages de git)%
\begin{longtable}{c c c c c c c}%
\hline%
Cours&TD&TP&Projet&Eval&Personnel&Total\\%
nan h&nan h&nan h&nan h&nan h&nan h&nan h\\%
\hline%
\end{longtable}%
\subsection{Méthodes et outils devops}%
\label{subsec:Mthodesetoutilsdevops}%

%
En s'appuyant sur des connaissances de base en développement et construction de logiciel, la matière prolonge ces compétences en y ajoutant l'intégration continue, la conteneurisation et le déploiement dans le cloud.%
\begin{longtable}{c c c c c c c}%
\hline%
Cours&TD&TP&Projet&Eval&Personnel&Total\\%
nan h&nan h&nan h&nan h&nan h&nan h&nan h\\%
\hline%
\end{longtable}%
\subsection{Parallélisation de données}%
\label{subsec:Paralllisationdedonnes}%

%
Cette matière sensibilise à l'apport du parallélisme pour la performance de traitement sur des données. La mise en correspondance entre "l'architecture" des problèmes et celles des infrastructures est discutée, ainsi que l'écriture d'algorithmes mettant en oeuvre ces opérateurs à parallélisation de données. %
\begin{longtable}{c c c c c c c}%
\hline%
Cours&TD&TP&Projet&Eval&Personnel&Total\\%
nan h&nan h&nan h&nan h&nan h&nan h&nan h\\%
\hline%
\end{longtable}%
\subsection{\textit{Total UE :}}%
\label{subsec:textitTotalUE}%

%
\begin{longtable}{c c c c}%
\hline%
Travail maquette&Travail personel&Travail total&Crédits ECTS\\%
nan h&nan h&nan h&5\\%
\hline%
\end{longtable}