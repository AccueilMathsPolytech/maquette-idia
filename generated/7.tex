\section{UE Mathématiques appliquées}%
\label{sec:UEMathmatiquesappliques}%
A COMPLETER%
\subsection{Algèbre linéaire}%
\label{subsec:Algbrelinaire}%

%
L'élément constitutif Algèbre linéaire vise à rendre les élèves capables d'identifier, modéliser et  utiliser un système linéaire à plusieurs variables. Via une série de problèmes (simulations de processus, codes correcteurs linéaires, algorithme de Gram{-}Schmitt, moindres carrés multivariés), la matière amène à manipuler les sytèmes linéaires, le calcul matriciel et ses principales opérations, les espaces vectoriels, leur extension euclidienne). La mise en pratique sur ces problèmes amène à étudier et pratiquer la manipulation informatique efficace des données et calcul en python avec Numpy.%
\begin{longtable}{c c c c c c c}%
\hline%
Cours&TD&TP&Projet&Eval&Personnel&Total\\%
nan h&nan h&nan h&nan h&nan h&nan h&nan h\\%
\hline%
\end{longtable}%
\subsection{Statistiques et probabilités}%
\label{subsec:Statistiquesetprobabilits}%

%
L'élément constitutif Statistiques et probabilités étudie les concepts de base de la théorie des probabilités et les distributions les plus courantes, pour permettre la modélisation et la résolution des problèmes réels ou théoriques. En complément, il introduit au raisonnement statistique et à la prise en compte de l'aléa en situation de décision et permet de s’initier aux étapes principales d'une démarche statistique (décrire, estimer, tester).%
\begin{longtable}{c c c c c c c}%
\hline%
Cours&TD&TP&Projet&Eval&Personnel&Total\\%
nan h&nan h&nan h&nan h&nan h&nan h&nan h\\%
\hline%
\end{longtable}%
\subsection{Projet traitement statistique de la donnée}%
\label{subsec:Projettraitementstatistiquedeladonne}%

%
A COMPLETER%
\begin{longtable}{c c c c c c c}%
\hline%
Cours&TD&TP&Projet&Eval&Personnel&Total\\%
nan h&nan h&nan h&nan h&nan h&nan h&nan h\\%
\hline%
\end{longtable}%
\subsection{\textit{Total UE :}}%
\label{subsec:textitTotalUE}%

%
\begin{longtable}{c c c c}%
\hline%
Travail maquette&Travail personel&Travail total&Crédits ECTS\\%
nan h&nan h&nan h&5\\%
\hline%
\end{longtable}