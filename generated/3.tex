\section{UE Architectures et systèmes informatiques}%
\label{sec:UEArchitecturesetsystmesinformatiques}%
Cette UE introduit les éléments techniques fondamentaux pour appréhender les architectures informatiques, d’un point de vue système d’exploitation, de communication en réseau et de stockage des données.%
\subsection{Systèmes informatiques}%
\label{subsec:Systmesinformatiques}%

%
L'élément constitutif Systèmes Informatiques a pour objet de présenter les principes des systèmes d'exploitation. Ces systèmes logiciels forment le cadre dans lequel se construisent et s'exécutent les logiciels applicatifs les plus divers, dont les traitements et bases de données. En particulier, dès qu'on conçoit ou on exploite du logiciel où la souplesse d'exploitation de l'infrastructure (par ex. cloud) ou la performance (calcul, interrogation) sont des questions, une compréhension des systèmes d'exploitation est précieuse. Dans ce domaine, cette matière en vise une première compréhension théorique (architecture des machines, machine de Turing, circuits logiques) et une première maîtrise pratique (gestion de fichiers, droits, processus) permettant ensuite un travail autonome en ligne de commande sur les machines (en développement logiciel, bases de données,....)%
\begin{longtable}{c c c c c c c}%
\hline%
Cours&TD&TP&Projet&Eval&Personnel&Total\\%
6.0 h&6.0 h&18.0 h&0.0 h&1.5 h&30.0 h&61.5 h\\%
\hline%
\end{longtable}%
\subsection{Réseaux}%
\label{subsec:Rseaux}%

%
L'élément constitutif Réseaux a pour objet d’introduire les notions fondamentales théoriques et pratiques sur les environnements informatiques composés de multiples machines communicantes. Ces systèmes permettent non seulement d'échanger des informations, mais aussi d'offrir des services (stockage, calcul) tels une machine abstraite unique, performante et souple d'usage (par ex. de type cloud), dont par exemple les applications intensifs en données ont besoin. Toutefois, la communication et la collaboration entre les machines soulèvent de nombreux problèmes étudiés dans les disciplines du réseau et des systèmes répartis. La matière conduit un double objectif : d'un part, rendre autonomes et éclairés en pratique les élèves pour des opérations quotidiennes de toute informaticien.ne connecté au réseau ; d'autre part, avoir une première compréhension des enjeux et les mécanismes impliqués dans l'exploitation des systèmes multi{-}machines.%
\begin{longtable}{c c c c c c c}%
\hline%
Cours&TD&TP&Projet&Eval&Personnel&Total\\%
3.5 h&3.5 h&7.5 h&0.0 h&1.5 h&15.0 h&31.0 h\\%
\hline%
\end{longtable}%
\subsection{Modèle et langage relationnels}%
\label{subsec:Modleetlangagerelationnels}%

%
L'élément constitutif Modèle relationnel introduit le modèle mathématique et les langages d’interrogation des bases de données relationnelles, à savoir le calcul et l’algèbre relationnels, ainsi que leur contre{-}partie pratique, le langage SQL. En outre, sont abordées la modélisation et la normalisation de schémas relationnels à l’aide de la notion de dépendance fonctionnelle. Ainsi, la matière apporte l'essentiel pour utiliser efficacement une base de données relationnelle. La compréhension des fonctionnnements internes du système est traitée dans des UE ultérieures.%
\begin{longtable}{c c c c c c c}%
\hline%
Cours&TD&TP&Projet&Eval&Personnel&Total\\%
7.5 h&7.5 h&15.0 h&0.0 h&1.5 h&30.0 h&61.5 h\\%
\hline%
\end{longtable}%
\subsection{\textit{Total UE :}}%
\label{subsec:textitTotalUE}%

%
\begin{longtable}{c c c c}%
\hline%
Travail maquette&Travail personel&Travail total&Crédits ECTS\\%
79.0 h&75.0 h&154.0 h&5\\%
\hline%
\end{longtable}