\section{UE Humanités S6 }%
\label{sec:UEHumanitsS6}%
A COMPLETER%
\subsection{Anglais {-} communication orale et outils linguistiques}%
\label{subsec:Anglais{-}communicationoraleetoutilslinguistiques}%

%
A COMPLETER%
\begin{longtable}{c c c c c c c}%
\hline%
Cours&TD&TP&Projet&Eval&Personnel&Total\\%
nan h&nan h&nan h&nan h&nan h&nan h&nan h\\%
\hline%
\end{longtable}%
\subsection{Simulation de gestion d'entreprise}%
\label{subsec:Simulationdegestiondentreprise}%

%
Appréhender le marketing, la stratégie d'entreprise et la gestion d'entreprise de façon ludique sur la base d'une mise en application simulant la gestion d'entreprise sur plusieurs années, le tout dans un univers concurrentiel. {-} Objectif 1 : acquérir les bases du marketing et de la gestion\newline%
{-} Objectif 2 : Mettre en application les éléments théoriques sur la base de la simulation dans laquelle il est demandé, tout au long de la simulation, de rendre des calculs précis et de rendre compte de la stratégie déployée\newline%
{-} Objectif 3 : Savoir rendre compte de manière synthétique de l’expérience vécue au sein d’un groupe\newline%
{-} Objectif 4 : savoir travailler en groupe et prendre en compte les divergences et les avis de chacun.%
\begin{longtable}{c c c c c c c}%
\hline%
Cours&TD&TP&Projet&Eval&Personnel&Total\\%
nan h&nan h&nan h&nan h&nan h&nan h&nan h\\%
\hline%
\end{longtable}%
\subsection{Enjeux de société et entreprise S6}%
\label{subsec:EnjeuxdesocitetentrepriseS6}%

%
Le module « Enjeux de société et entreprises » vise à aborder des méthodologies d’analyse via la réalisation d’une étude collective (en groupe de 4 ou 5 apprentis, sur les deux premières années) qui traite des problématiques en lien avec leur secteur d’activité et notamment autour des dimensions souvent occultées en entreprise et/ou qui n’ont pas le temps d’être travaillées dans la formation. A titre d’exemple, les études peuvent porter sur la place des femmes dans le secteur informatique ou industriel, les représentations du bois sur le marché du bâtiment, le discours médiatique autour de l’« IA », du rapport « homme{-}machine » dans l‘industrie, les croyances et les pratiques professionnelles en termes de sécurité des données numériques. La mutualisation des contextes d’entreprises et des expériences des élèves doit permettre le partage, la confrontation et une certaine montée en généralité des résultats.%
\begin{longtable}{c c c c c c c}%
\hline%
Cours&TD&TP&Projet&Eval&Personnel&Total\\%
nan h&nan h&nan h&nan h&nan h&nan h&nan h\\%
\hline%
\end{longtable}%
\subsection{Sciences sociales appliquées au travail S6}%
\label{subsec:SciencessocialesappliquesautravailS6}%

%
Cette matière vis à faire acquérir une démarche rationnelle de questionnement sur le travail et l'homme et une méthodologie de recueil de données adaptée à ce questionnement. S'approprier les savoirs relatifs à une pratique liée au travail humain. Faire se rejoindre "pratiques" et "théories" à partir de l'expérience professionnelle des élèves ingénieurs en apprentissage (analyse de la pratique). Transformer ces savoirs en savoirs{-}faire professionnels. Ce semestre, on s'intéressera en particulier à la fonction management, la compréhension des organisations, au changement et à l'innovation, à l'animation de réunion.%
\begin{longtable}{c c c c c c c}%
\hline%
Cours&TD&TP&Projet&Eval&Personnel&Total\\%
nan h&nan h&nan h&nan h&nan h&nan h&nan h\\%
\hline%
\end{longtable}%
\subsection{Projet de séjour à l'international S6}%
\label{subsec:ProjetdesjourlinternationalS6}%

%
Cette matière apporte des compétences en vue de faciliter la recherche du séjour à l'étranger des apprentis (prospection, présentation de soi en perspectives internationale) et d'en comprendre les apports visés pour mieux s'y préparer (expérience et travail interculturels).%
\begin{longtable}{c c c c c c c}%
\hline%
Cours&TD&TP&Projet&Eval&Personnel&Total\\%
nan h&nan h&nan h&nan h&nan h&nan h&nan h\\%
\hline%
\end{longtable}%
\subsection{Analyse de la pratique S6}%
\label{subsec:AnalysedelapratiqueS6}%

%
Cette matière vise à permettre aux apprentis de passer d’une position «d’étudiant» à une position de «professionnel», grâce à une réflexion sur leurs modes et méthodologies d’apprentissage; une identification des pratiques efficientes;un échange entre pairs; une mise en lien des deux lieux de formation que sont l’école et l’entreprise d’accueil. On s'intéressera en particulier à l'identification des bonnes pratiques d'intégration de l’apprenti en entreprise.%
\begin{longtable}{c c c c c c c}%
\hline%
Cours&TD&TP&Projet&Eval&Personnel&Total\\%
nan h&nan h&nan h&nan h&nan h&nan h&nan h\\%
\hline%
\end{longtable}%
\subsection{Total UE :}%
\label{subsec:TotalUE}%

%
\begin{longtable}{c c c c}%
\hline%
Travail maquette&Travail personel&Travail total&Crédits ECTS\\%
nan h&nan h&nan h&5\\%
\hline%
\end{longtable}