\section{UE Humanités S7}%
\label{sec:UEHumanitsS7}%
A COMPLETER%
\subsection{Sciences sociales appliquées au travail S7}%
\label{subsec:SciencessocialesappliquesautravailS7}%

%
Cette matière vis à faire acquérir une démarche rationnelle de questionnement sur le travail et l'homme et une méthodologie de recueil de données adaptée à ce questionnement. S'approprier les savoirs relatifs à une pratique liée au travail humain. Faire se rejoindre "pratiques" et "théories" à partir de l'expérience professionnelle des élèves ingénieurs en apprentissage (analyse de la pratique). Transformer ces savoirs en savoirs{-}faire professionnels. %
\begin{longtable}{c c c c c c c}%
\hline%
Cours&TD&TP&Projet&Eval&Personnel&Total\\%
nan h&nan h&nan h&nan h&nan h&nan h&nan h\\%
\hline%
\end{longtable}%
\subsection{Economie d'entreprise S7}%
\label{subsec:EconomiedentrepriseS7}%

%
A COMPLETER%
\begin{longtable}{c c c c c c c}%
\hline%
Cours&TD&TP&Projet&Eval&Personnel&Total\\%
nan h&nan h&nan h&nan h&nan h&nan h&nan h\\%
\hline%
\end{longtable}%
\subsection{Analyse de la pratique S7}%
\label{subsec:AnalysedelapratiqueS7}%

%
Cette matière vise à permettre aux apprentis de passer d’une position «d’étudiant» à une position de «professionnel», grâce à une réflexion sur leurs modes et méthodologies d’apprentissage; une identification des pratiques efficientes;un échange entre pairs; une mise en lien des deux lieux de formation que sont l’école et l’entreprise d’accueil.%
\begin{longtable}{c c c c c c c}%
\hline%
Cours&TD&TP&Projet&Eval&Personnel&Total\\%
nan h&nan h&nan h&nan h&nan h&nan h&nan h\\%
\hline%
\end{longtable}%
\subsection{Projet de séjour à l'international S7}%
\label{subsec:ProjetdesjourlinternationalS7}%

%
A COMPLETER%
\begin{longtable}{c c c c c c c}%
\hline%
Cours&TD&TP&Projet&Eval&Personnel&Total\\%
nan h&nan h&nan h&nan h&nan h&nan h&nan h\\%
\hline%
\end{longtable}%
\subsection{Anglais {-} corporate culture }%
\label{subsec:Anglais{-}corporateculture}%

%
A COMPLETER%
\begin{longtable}{c c c c c c c}%
\hline%
Cours&TD&TP&Projet&Eval&Personnel&Total\\%
nan h&nan h&nan h&nan h&nan h&nan h&nan h\\%
\hline%
\end{longtable}%
\subsection{Enjeux de société et entreprise S7}%
\label{subsec:EnjeuxdesocitetentrepriseS7}%

%
Le module « Enjeux de société et entreprises » vise à aborder des méthodologies d’analyse via la réalisation d’une étude collective (en groupe de 4 ou 5 apprentis, sur les deux premières années) qui traite des problématiques en lien avec leur secteur d’activité et notamment autour des dimensions souvent occultées en entreprise.%
\begin{longtable}{c c c c c c c}%
\hline%
Cours&TD&TP&Projet&Eval&Personnel&Total\\%
nan h&nan h&nan h&nan h&nan h&nan h&nan h\\%
\hline%
\end{longtable}%
\subsection{Total UE :}%
\label{subsec:TotalUE}%

%
\begin{longtable}{c c c c}%
\hline%
Travail maquette&Travail personel&Travail total&Crédits ECTS\\%
nan h&nan h&nan h&5\\%
\hline%
\end{longtable}