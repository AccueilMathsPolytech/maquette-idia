\section{UE Données et connaissances}%
\label{sec:UEDonnesetconnaissances}%
A COMPLETER%
\subsection{Entrepôts de données}%
\label{subsec:Entreptsdedonnes}%

%
L'élément constitutif SQL avancé et entrepôts de données aborde la modélisation de bases de données multidimensionnelles, conçues pour héberger toutes les données d’une organisation en vue de leur exploitation dans des systèmes décisionnels. En outre, nous abordons les mécanismes complexes et néanmoins utiles du langage SQL, tels que les requêtes récursives ou les requêtes analytiques, la division relationnelle ou encore les requêtes imbriquées.%
\begin{longtable}{c c c c c c c}%
\hline%
Cours&TD&TP&Projet&Eval&Personnel&Total\\%
nan h&nan h&nan h&nan h&nan h&nan h&nan h\\%
\hline%
\end{longtable}%
\subsection{Processus de Business Intelligence}%
\label{subsec:ProcessusdeBusinessIntelligence}%

%
A COMPLETER%
\begin{longtable}{c c c c c c c}%
\hline%
Cours&TD&TP&Projet&Eval&Personnel&Total\\%
nan h&nan h&nan h&nan h&nan h&nan h&nan h\\%
\hline%
\end{longtable}%
\subsection{Total UE :}%
\label{subsec:TotalUE}%

%
\begin{longtable}{c c c c}%
\hline%
Travail maquette&Travail personel&Travail total&Crédits ECTS\\%
nan h&nan h&nan h&5\\%
\hline%
\end{longtable}