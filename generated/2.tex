\section{UE Informatique fondamentale}%
\label{sec:UEInformatiquefondamentale}%
Cette UE introduit les fondements théoriques indispensables à la modélisation et la résolution de problèmes complexes en machine%
\subsection{Théorie des graphes}%
\label{subsec:Thoriedesgraphes}%

%
L'élément constitutif Théorie des graphes présente les représentations mathématiques des graphes et étudie leurs nombreuses propriétés. Le cours aborde ensuite les principes de la modélisation de problèmes réels sous la forme de graphes, pour enfin développer quelques algorithmes fondamentaux sur les graphes (recherche du plus court chemin, arbre couvrant, etc.).%
\begin{longtable}{c c c c c c c}%
\hline%
Cours&TD&TP&Projet&Eval&Personnel&Total\\%
10.0 h&10.0 h&0.0 h&0.0 h&1.5 h&20.0 h&41.5 h\\%
\hline%
\end{longtable}%
\subsection{Logique mathématique}%
\label{subsec:Logiquemathmatique}%

%
L'élément constitutif Logique mathématique commence par rappeler les principes de la théorie des ensembles et celle des ordres, pour ensuite introduire le cœur de cet élément constitutif, les logiques classiques : logique des propositions et logique des prédicats. Les compétences d'apparence un peu formelle en logique permettent ensuite une compréhension des mécanismes performants de modélisation et d'interrogation des structures et bases de données. Elles sont également utiles pour voir la tâche de programmation à un niveau d'abstraction élevé : la spécification, la vérification, les langages. Enfin, la logique est une des facettes de l'IA (encodage des connaissances, raisonnement, explicabilité). %
\begin{longtable}{c c c c c c c}%
\hline%
Cours&TD&TP&Projet&Eval&Personnel&Total\\%
8.0 h&12.0 h&0.0 h&0.0 h&1.5 h&20.0 h&41.5 h\\%
\hline%
\end{longtable}%
\subsection{Automates et probabilités}%
\label{subsec:Automatesetprobabilits}%

%
L'élément constitutif Automates et probabilités introduit la théorie des langages et les grammaires formelles comme clé de compréhension de la programmation. L’étude des langages réguliers offre en outre des outils précieux pour l’analyse d’algorithme et la preuve de programmes. C’est également le lieu de la présentation d’une construction puissante, les expressions régulières. Les automates finis, contre{-}partie pratique des langages réguliers, constituent un outil de premier choix dans d’innombrables champs d’application (conception de programmes informatiques et de protocoles de communication, linguistique, biologie, intelligence artificielle, etc.).%
\begin{longtable}{c c c c c c c}%
\hline%
Cours&TD&TP&Projet&Eval&Personnel&Total\\%
10.0 h&10.0 h&0.0 h&0.0 h&1.5 h&20.0 h&41.5 h\\%
\hline%
\end{longtable}%
\subsection{\textit{Total UE :}}%
\label{subsec:textitTotalUE}%

%
\begin{longtable}{c c c c}%
\hline%
Travail maquette&Travail personel&Travail total&Crédits ECTS\\%
64.5 h&60.0 h&124.5 h&5\\%
\hline%
\end{longtable}