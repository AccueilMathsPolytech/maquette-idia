\section{UE Logiciel}%
\label{sec:UELogiciel}%
Cette UE, à la suite du semestre 5, vise à compléter la culture et le caractère opérationnel des apprentis sur l'aspect logiciel. Elle introduit les paradigmes de développement logiciel modernes au travers de deux modèles~: la programmation à objets d’une part, et les technologies liées au développement sur le web d’autre part.%
\subsection{Programmation objet en Java}%
\label{subsec:ProgrammationobjetenJava}%

%
Cette matière vise à enrichir la connaissance et la pratique en matière de conception et développement logiciel, au moyen de la programmation orientée objet. Les procédés associés (encapsulation/modularité, sous{-}typage et héritage/généricité,...) permettent en effet d'améliorer le qualité du logiciel, par rapport à un langage impératif supposé préalablement connu par des étudiants. Le langage support utilisé est Java, car il est largement utilisé dans l'industrie informatique. Les questions de typage, surcharge, liaisons statique et dynamique sont étudiées.%
\begin{longtable}{c c c c c c c}%
\hline%
Cours&TD&TP&Projet&Eval&Personnel&Total\\%
nan h&nan h&nan h&nan h&nan h&nan h&nan h\\%
\hline%
\end{longtable}%
\subsection{Technologies web}%
\label{subsec:Technologiesweb}%

%
Cette matière vise à fournir des bases techniques pour développer et déployer du logiciel sur le web. Pour cela, HTML, CSS et JavaScript sont étudiés (JavaScript objet{-}par prototype, fonctionnel, asynchronisme, défensif), puis les mécanismes de service : HTTP, websocket, serveurs NodeJS, services web et protocoles.  %
\begin{longtable}{c c c c c c c}%
\hline%
Cours&TD&TP&Projet&Eval&Personnel&Total\\%
nan h&nan h&nan h&nan h&nan h&nan h&nan h\\%
\hline%
\end{longtable}%
\subsection{Total UE :}%
\label{subsec:TotalUE}%

%
\begin{longtable}{c c c c}%
\hline%
Travail maquette&Travail personel&Travail total&Crédits ECTS\\%
nan h&nan h&nan h&5\\%
\hline%
\end{longtable}