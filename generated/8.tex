\section{UE Systèmes d'information}%
\label{sec:UESystmesdinformation}%
A COMPLETER%
\subsection{Conception des systèmes d'information}%
\label{subsec:Conceptiondessystmesdinformation}%

%
L’élément constitutif Conception de systèmes d’information introduit différents principes, outils et langages de modélisation tels que UML ou les diagrammes Entités{-}Associations, dont la traduction vers le modèle relationnel est très précisément étudiée. Une palette riche de diagrammes est présentée, pour proposer des points de vue complémentaires dans l’activité de modélisation de SI : état{-}transition, flux, séquences, etc.%
\begin{longtable}{c c c c c c c}%
\hline%
Cours&TD&TP&Projet&Eval&Personnel&Total\\%
nan h&nan h&nan h&nan h&nan h&nan h&nan h\\%
\hline%
\end{longtable}%
\subsection{Traitement de requêtes}%
\label{subsec:Traitementderequtes}%

%
Evaluation et optimisation de requête, algorithmes de tri et de jointure. Indexation (structures de données) et transcription des requêtes aux bases de données en algorithmes efficaces. Organisation physique et performance.%
\begin{longtable}{c c c c c c c}%
\hline%
Cours&TD&TP&Projet&Eval&Personnel&Total\\%
nan h&nan h&nan h&nan h&nan h&nan h&nan h\\%
\hline%
\end{longtable}%
\subsection{Systèmes transactionnels}%
\label{subsec:Systmestransactionnels}%

%
Cette manière approfondit le domaine des bases de données en s'intéressant aux transactions, c’est{-}à{-}dire à l'analyse des problèmes soulevés par l'accès potentiellement concurrents d'accès aux bases de données. Les mécanismes et solutions sont exposés en ayant recours au contrôle de concurrence, l'isolation, la sérialisabilité, les verrous.%
\begin{longtable}{c c c c c c c}%
\hline%
Cours&TD&TP&Projet&Eval&Personnel&Total\\%
nan h&nan h&nan h&nan h&nan h&nan h&nan h\\%
\hline%
\end{longtable}%
\subsection{Total UE :}%
\label{subsec:TotalUE}%

%
\begin{longtable}{c c c c}%
\hline%
Travail maquette&Travail personel&Travail total&Crédits ECTS\\%
nan h&nan h&nan h&5\\%
\hline%
\end{longtable}