\section{UE Humanités}%
\label{sec:UEHumanits}%
A COMPLETER%
\subsection{Sciences Sociales Appliquées au Travail S5}%
\label{subsec:SciencesSocialesAppliquesauTravailS5}%

%
Cette matière vis à faire acquérir une démarche rationnelle de questionnement sur le travail et l'homme et une méthodologie de recueil de données adaptée à ce questionnement. S'approprier les savoirs relatifs à une pratique liée au travail humain. Faire se rejoindre "pratiques" et "théories" à partir de l'expérience professionnelle des élèves ingénieurs en apprentissage (analyse de la pratique). Transformer ces savoirs en savoirs{-}faire professionnels. Ce semestre, on s'intéressera en particulier à observer et questionner le travail et construire une problématique.%
\begin{longtable}{c c c c c c c}%
\hline%
Cours&TD&TP&Projet&Eval&Personnel&Total\\%
nan h&28.0 h&10.0 h&nan h&nan h&5.0 h&nan h\\%
\hline%
\end{longtable}%
\subsection{Anglais}%
\label{subsec:Anglais}%

%
A COMPLETER%
\begin{longtable}{c c c c c c c}%
\hline%
Cours&TD&TP&Projet&Eval&Personnel&Total\\%
nan h&22.5 h&nan h&nan h&nan h&5.0 h&nan h\\%
\hline%
\end{longtable}%
\subsection{Economie}%
\label{subsec:Economie}%

%
A COMPLETER (en attente retour enseignant)%
\begin{longtable}{c c c c c c c}%
\hline%
Cours&TD&TP&Projet&Eval&Personnel&Total\\%
nan h&20.0 h&nan h&nan h&nan h&nan h&nan h\\%
\hline%
\end{longtable}%
\subsection{Analyse de la pratique S5}%
\label{subsec:AnalysedelapratiqueS5}%

%
Cette matière vise à permettre aux apprentis de passer d’une position «d’étudiant» à une position de «professionnel», grâce à une réflexion sur leurs modes et méthodologies d’apprentissage; une identification des pratiques efficientes;un échange entre pairs; une mise en lien des deux lieux de formation que sont l’école et l’entreprise d’accueil. On s'intéressera en particulier à l'identification des bonnes pratiques d'intégration de l’apprenti en entreprise.%
\begin{longtable}{c c c c c c c}%
\hline%
Cours&TD&TP&Projet&Eval&Personnel&Total\\%
nan h&nan h&nan h&nan h&nan h&nan h&nan h\\%
\hline%
\end{longtable}%
\subsection{Total UE :}%
\label{subsec:TotalUE}%

%
\begin{longtable}{c c c c}%
\hline%
Travail maquette&Travail personel&Travail total&Crédits ECTS\\%
nan h&nan h&nan h&5\\%
\hline%
\end{longtable}