\section{UE Programmation et algorithmes}%
\label{sec:UEProgrammationetalgorithmes}%
Cette UE aborde les principes et outils de la programmation logicielle, à la fois du point de vue de la construction de programmes pour la résolution de problèmes, et également par le prisme des méthodes et outils de mesure indispensables à l’appropriation des bonnes pratiques en matière de développement logiciel.%
\subsection{Algorithmes et structures de données}%
\label{subsec:Algorithmesetstructuresdedonnes}%

%
L'élément constitutif Algorithmes et structures de données offre le cadre théorique et pratique de la programmation informatique. Il y est question des mécanismes algorithmiques de résolution de problèmes et d’analyse d’algorithmes. L’enjeu consiste à produire du code informatique efficace pour modéliser et résoudre une large gamme de problèmes. Le langage support de cet apprentissage est Python. Une part non négligeable de l’enseignement est dévolue à l’adoption de bonnes pratiques de la programmation.%
\begin{longtable}{c c c c c c c}%
\hline%
Cours&TD&TP&Projet&Eval&Personnel&Total\\%
10.0 h&10.0 h&20.0 h&0.0 h&1.5 h&30.0 h&71.5 h\\%
\hline%
\end{longtable}%
\subsection{Introduction au développement logiciel}%
\label{subsec:Introductionaudveloppementlogiciel}%

%
Prolongeant les premiers enseignements en algorithmique et python, l'élément constitutif Introduction au développement logiciel présente et justifie des premières méthodes organisationnelles et technologiques pour le développement logiciel. Cette matière vise des compétences complémentaires relatives à la qualité du processus de programmation, en cadre collaboratif, et à la qualité du code produit. Ayant exposé les raisons et principes du génie logiciel, une initiation théorique et pratique est fournie, qui couvre la gestion de version (git), le test unitaire, l'intégration continue et l'analyse statique de code.%
\begin{longtable}{c c c c c c c}%
\hline%
Cours&TD&TP&Projet&Eval&Personnel&Total\\%
3.5 h&3.5 h&7.0 h&0.0 h&1.5 h&20.0 h&35.5 h\\%
\hline%
\end{longtable}%
\subsection{Projet de développement logiciel}%
\label{subsec:Projetdedveloppementlogiciel}%

%
L'élément constitutif Projet de développement logiciel offre un cadre pratique pour s’exercer à la résolution de problèmes complexes (scénario d'optimisation combinatoire à traiter par des heuristiques spécifiques au scénario) par la construction de programmes informatiques en Python. Le problème comporte une sensibilisation à la recherche de solutions de complexité acceptables en pratique. Ce travail de groupe (mise en pratique systématique de git) est réalisé dans le cadre d’un concours de programmation de type Google Hashcode.%
\begin{longtable}{c c c c c c c}%
\hline%
Cours&TD&TP&Projet&Eval&Personnel&Total\\%
2.5 h&2.5 h&0.0 h&50.0 h&1.5 h&0.0 h&56.5 h\\%
\hline%
\end{longtable}%
\subsection{\textit{Total UE :}}%
\label{subsec:textitTotalUE}%

%
\begin{longtable}{c c c c}%
\hline%
Travail maquette&Travail personel&Travail total&Crédits ECTS\\%
113.5 h&50.0 h&163.5 h&5\\%
\hline%
\end{longtable}