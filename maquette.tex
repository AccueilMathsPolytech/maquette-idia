% LaTeX Template for short student reports.
% Citations should be in bibtex format and go in references.bib
\documentclass[a4paper, 11pt]{article}
\usepackage[top=3cm, bottom=3cm, left = 2cm, right = 2cm]{geometry} 
\geometry{a4paper} 
\usepackage[utf8]{inputenc}
\usepackage{xcolor}
\usepackage[french]{babel}
\usepackage{textcomp}
\usepackage{graphicx} 
\usepackage{amsmath,amssymb}  
\usepackage{bm}  
\usepackage[pdftex,bookmarks,colorlinks,breaklinks]{hyperref}  
\hypersetup{linkcolor=black,citecolor=black,filecolor=black,urlcolor=black} % black links, for printed output
\usepackage{memhfixc} 
\usepackage{pdfsync}  
\usepackage{fancyhdr}
\usepackage{multirow}
\usepackage{array}
\usepackage{longtable}
\pagestyle{fancy}

\title{Ingénieur spécialité informatique \\ parcours Ingénierie de la donnée et de l'intelligence articifielle \\ (par la voie de l'apprentissage))}
\date{}

\begin{document}


\begin{titlepage}
    \begin{centering}
    \newcommand{\HRule}{\rule{\linewidth}{0.5mm}}

    %\includegraphics[scale=0.1]{ponts.png} \\[1cm]
    \HRule \\[0.4cm]
    { \huge \bfseries Ingénieur spécialité informatique (INFO)\linebreak 
    \vspace*{1cm} \huge parcours 
    \linebreak par la voie de l'apprentissage \vspace*{1cm} \linebreak
     Ingénierie de la donnée 
     \linebreak et de l'intelligence articifielle (IDIA) 
     \linebreak \vspace*{1cm} \\[0.15cm] }
    \HRule \\[1.5cm]
    \Huge Polytech Nantes
    \\[1cm]
    \includegraphics[scale=0.4]{polytech-nantes.png}
    \end{centering}

\vfill

Document compilé le \today.
\end{titlepage}

\vfill

Ce document présente le programme de formation se déroulant sur le cycle ingénieur (années bac+3,bac+5,bac+5), tel que prévu pour l'année 2022-2023. Quelques modifications (interversions de matières/années) prévues pour l'année 2023-2024 sont incluses pour anticiper sur la version prévisionnelle stabilisée future.

Ce programme ne décrit que la partie de la formation se déroulant à l'école (60 semaines sur 3 ans). Bien entendue, la partie se déroulant en entreprise contribue aussi à la formation de l'apprenti.e.

Chaque semestre est composé d'Unités d'Enseignements (UE), elles-mêmes composées de matières (parfois nommés "éléments constitutifs" (d'UE)). L'organisation en Unités d'Enseignement correspond aux grands domaines de compétences de la formation. La validation d'une UE attribue à l'élève un nombre de crédits dits ECTS (European Credit Transfert System) associé à l'UE. Une année est composée de deux semestres, chacun composé de 30 ECTS. Ainsi, le cycle ingénieur sur 3 ans apporte 180 ECTS. Le niveau de compétence de l'élève, lui, sera évalué par un grade associé à cette UE (une lettre entre A et F). Les crédits ECTS associés à une UE sont des indicateurs du temps de travail total de l'élève (présentiel école et personnel) associé à cette UE. Pour les UE en formation à l'école, la norme est d'environ 25h de travail par crédit.  

Maintenant j'ajoue des remarques dans ce texte.
 
\pagebreak

\tableofcontents
\pagebreak 
\part{Vue synthétique}
\include{"generated/semestre5"} \pagebreak
\include{"generated/semestre6"} \pagebreak
\include{"generated/semestre7"} \pagebreak
\include{"generated/semestre8"} \pagebreak

\part{3eme année - semestre 5}
\include{"generated/1"}
\include{"generated/2"}
\include{"generated/3"}
\include{"generated/4"}
\include{"generated/5"}
\part{3eme année - semestre 6}
\include{"generated/6"}
\include{"generated/7"}
\include{"generated/8"}
\include{"generated/9"}
\include{"generated/10"}
\part{4eme année - semestre 7}
\include{"generated/11"}
\include{"generated/12"}
\include{"generated/13"}
\include{"generated/14"}
\include{"generated/15"}
\part{4eme année - semestre 8}
\include{"generated/16"}
\include{"generated/17"}
\include{"generated/18"}
\include{"generated/19"}
\include{"generated/20"}
\part{5eme année - semestre 9}



\bibliographystyle{abbrv}
% \bibliography{references}  % need to put bibtex references in references.bib 
\end{document}

